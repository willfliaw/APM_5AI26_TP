\chapter{Introduction} \label{chap:intro}

In recent years, kernel methods have become a cornerstone of machine learning, especially for tasks involving non-linear data structures. Kernels enable linear algorithms to operate in transformed feature spaces, allowing for the efficient handling of complex relationships within data. Time-series data, with its inherent sequential dependencies, is a particularly challenging domain for machine learning, demanding specialized methods to capture temporal patterns and underlying dynamics, and arguably one of the most important data types in the moders era \cite{hamilton1994, strogatz2018, zhang2017, zeroual2020}. As traditional kernel approaches often struggle with the complexity of time-series data, recent research has explored innovative kernel designs tailored specifically for this data type.

The selected paper, ``Time Series Kernels based on Nonlinear Vector AutoRegressive Delay Embeddings'' \cite{felice2023}, addresses a major challenge in time-series kernel design by introducing a new kernel based on Nonlinear Vector AutoRegressive (NVAR) delay embeddings. The proposed NVAR kernel draws on the principles of reservoir computing, adapting them into a more interpretable and computationally efficient framework. Unlike standard reservoir computing-based kernels, which rely on recurrent structures and complex hyperparameter tuning, the NVAR kernel leverages non-recursive embeddings. This approach reduces the dependency on recurrent hyperparameters, making it better suited for classification tasks with small datasets, where deep learning techniques may not be feasible.

This project aims to conduct a comprehensive analysis of the NVAR kernel's methodology, positioning, and effectiveness in the broader landscape of kernel machines. By comparing the NVAR kernel with benchmarks in a select few datasets, we seek to evaluate its advantages and limitations in terms of both accuracy and computational efficiency. Furthermore, this study involves implementing and testing the NVAR kernel on a new dataset to assess its practical applicability and performance in real-world scenarios.

This report is structured as follows: First, we present an in-depth analysis of the NVAR kernel and its unique contributions to time-series classification. We then describe the methodology, experimental setup, followed by the results of testing the kernel on distinct datasets and a critical evaluation of the method's strengths and weaknesses. Finally, we discuss potential future directions for kernel design in time-series analysis and conclude with key insights drawn from this study.
