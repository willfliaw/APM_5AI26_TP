\chapter{Analysis} \label{chap:analysis}

\section{Novelty and Contributions}

Explain the novel aspects of the paper's proposed method, such as the use of NVAR for time-series kernel design.

Discuss how the NVAR kernel improves upon traditional reservoir computing-based kernels.

\section{Methodology}

Describe the methodology used in the paper, focusing on key elements such as kernel design, NVAR framework, and its application to time-series data.

Mention the significance of the kernel trick and how the paper builds on it.

\section{Comparison to State-of-the-Art}

Discuss how the paper positions its contribution against existing methods (e.g., reservoir computing, time-series classification techniques).

\section{Strengths}

Highlight the strengths of the paper, such as computational efficiency, scalability, and applicability to small datasets.

\section{Weaknesses}

Identify limitations in the approach, such as potential difficulties in hyperparameter tuning or specific contexts where the method may not perform as well (e.g., in high-dimensional datasets).
